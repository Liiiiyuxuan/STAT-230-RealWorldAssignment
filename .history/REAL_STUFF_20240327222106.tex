\documentclass[final]{beamer}
\usepackage[scale=1.24]{beamerposter} % Use the beamerposter package for laying out the poster

\usepackage{amsmath, amsthm, amssymb, latexsym}

% Define some colors
\definecolor{DarkBlue}{rgb}{0.1,0.1,0.5}
\definecolor{LightBlue}{rgb}{0.3,0.3,0.9}
\setbeamercolor{block title}{fg=LightBlue,bg=LightBlue}
\setbeamercolor{block body}{fg=black,bg=white}

% Title
\title{Fundamental Theorems in Mathematics}

% Author
\author{Author Name}

% Institution
\institute{Math Department, University X}

% Date
\date{\today}

\begin{document}
\begin{frame}[t]
\begin{columns}[t]

% First column
\begin{column}{.48\linewidth}
\begin{block}{Introduction}
    In some popular variations of poker such as Texas hold 'em, the most widespread poker variant overall,[3] a player uses the best five-card poker hand out of seven cards.
\end{block}

\begin{block}{Fundamental Theorem of Calculus}
The Fundamental Theorem of Calculus links the concept of the derivative of a function with the concept of the integral.
\[
\frac{d}{dx} \int_a^x f(t)\, dt = f(x)
\]
\end{block}
\end{column}

% Second column
\begin{column}{.48\linewidth}
\begin{block}{Pythagorean Theorem}
In a right-angled triangle, the square of the hypotenuse (the side opposite the right angle) is equal to the sum of the squares of the other two sides.
\[
a^2 + b^2 = c^2
\]
\end{block}

\begin{block}{Euler's Identity}
Euler's Identity is a special case of Euler's formula from complex analysis and is sometimes cited as an example of mathematical beauty.
\[
e^{i\pi} + 1 = 0
\]
\end{block}
\end{column}


\end{columns}
\end{frame}
\end{document}
