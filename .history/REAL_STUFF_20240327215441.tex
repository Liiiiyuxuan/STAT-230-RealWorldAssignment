In a 7-card poker game, calculating the probability of getting a Royal Flush involves considering that you're dealing with a larger set of cards from the deck, but the core approach to finding the probability remains similar. Here, instead of choosing 5 cards out of 52, you're choosing 7 cards. However, the condition for a Royal Flush still depends on having those specific 5 cards (Ace, King, Queen, Jack, and 10 of the same suit), with the other two cards being any of the remaining 47 cards in the deck. The general steps to derive the expression would involve:

1. **Total Number of Possible 7-Card Poker Hands**: This is calculated as the number of combinations of 52 cards taken 7 at a time, which is \(C(52, 7)\).

2. **Number of Ways to Get a Royal Flush in 7 Cards**: For a 7-card hand to contain a Royal Flush, it must contain the specific set of 5 cards (Ace, King, Queen, Jack, 10 of the same suit), with the other 2 cards being any of the remaining 47 cards in the deck. Since the Royal Flush can come from any of the four suits, and the two additional cards can be any other two cards from the remaining 47, the number of ways to get a Royal Flush in a 7-card hand is the product of the number of suits (4) and the number of combinations of the remaining 47 cards taken 2 at a time, which is \(C(47, 2)\).

3. **General Expression for the Probability**: The probability of getting a Royal Flush in a 7-card poker hand is the ratio of the number of ways to get a Royal Flush to the total number of possible 7-card hands. 

So, the general expression for the probability \(P(\text{Royal Flush in 7-card poker})\) is:

\[
P(\text{Royal Flush in 7-card poker}) = \frac{4 \times C(47, 2)}{C(52, 7)}
\]

This expression gives you the probability of being dealt a Royal Flush in a game of 7-card poker.

\documentclass[portrait]{Hylangtechposter}


\begin{document}

\printheader

\begin{center}
%% TITLE
\textbf{\bf\veryHuge\color{NavyBlue}I always thought there was something fundamentally wrong with the Universe \\[1.5cm]}

%% AUTHORS

\huge     First Author $^{\spadesuit}$ ~
    Second Author $^{\spadesuit}$ ~ 
    ThirdOne W. Different-Affiliation$^{\diamondsuit}$ ~ 
    F. Author $^{\spadesuit}$ \\[0.2cm] 
    Third Affiliation$^{\heartsuit}$ ~ 
    You Got It ByNow$^{\spadesuit}$ ~
    Is This the PI?$^{\spadesuit}$ \\[0.2cm]
\Large    $^{\spadesuit}$\texttt{\{firstname.lastname\}@helsinki.fi} \qquad $^{\diamondsuit}$\texttt{thirdone@nsecondaffiliation.com} \\
\Large   $^{\heartsuit}$\texttt{noname.3affiliation@somerandomuni.edu} 
 

 \end{center}

\vspace{2cm}

%%%% If you have an abstract:  uncomment %%%%
%%%%
% \color{Navy}
% \begin{abstract}

%   Te ancillae contentiones vix, ad partiendo patrioque inciderint
%   eum. Mea porro regione ex. Case iuvaret ocurreret quo at, legere
%   malorum indoctum cu his. Eros ubique mel in. At duo partem vidisse
%   intellegam. Equidem detraxit has ea. Phasellus imperdiet, tortor
%   vitae congue bibendum, felis enim sagittis lorem, et volutpat ante
%   orci sagittis mi. Morbi rutrum laoreet semper. Morbi accumsan enim
%   nec tortor consectetur non commodo nisi sollicitudin. Proin
%   sollicitudin. Pellentesque eget orci eros. Fusce ultricies, tellus
%   et pellentesque fringilla, ante massa luctus libero, quis tristique
%   purus urna nec nibh.
  
% \end{abstract}

% \vspace{1cm}

% As FONTES podem ser aumentadas para
% \large \Large \LARGE \huge \Huge \veryHuge \VeryHuge \VERYHuge

 \Large

\begin{multicols}{2} % begin two columns

  \color{black}
  
\section*{Introduction}
\lipsum[1]
\section*{Contributions}

% \color{DarkSlateGray} % DarkSlateGray color for the rest of the content

\begin{enumerate}
\item Lorem ipsum dolor sit amet, consectetur.
\item Nullam at mi nisl. Vestibulum est purus, ultricies cursus
  volutpat sit amet.
\item Praesent tortor libero, vulputate quis elementum a, iaculis.
\item Phasellus a quam mauris, non varius mauris. Fusce tristique,
  enim tempor varius porta, elit purus commodo velit, pretium mattis
  ligula nisl nec ante.
\end{enumerate}

\section{Experimental setup}

Sed ut perspiciatis, unde omnis iste natus error sit voluptatem
accusantium doloremque laudantium, totam rem aperiam eaque ipsa, quae
ab illo inventore veritatis et quasi architecto beatae vitae dicta
sunt, explicabo. Nemo enim ipsam voluptatem na Figura~1, quia voluptas
sit, aspernatur aut odit aut fugit, sed quia consequuntur magni
dolores eos.

\begin{center}\vspace{1cm}
\includegraphics[width=0.5\linewidth]{dummy_net-1.png}
\captionof{figure}{\color{Green} an important plot}
\end{center}\vspace{1cm}


\section{Experiments \& Methodology}

Fusce magna risus, molestie ut porttitor in, consectetur sed
mi. Vestibulum ante ipsum primis in faucibus orci luctus et ultrices
posuere cubilia Curae; Pellentesque consectetur blandit
pellentesque. Sed odio justo, viverra nec porttitor vel, lacinia a
nunc. Suspendisse pulvinar euismod arcu, sit amet accumsan enim
fermentum quis. In id mauris ut dui feugiat egestas. 


Nulla vel nisl sed mauris auctor mollis non sed. 

\begin{equation}
E = mc^{2}
\label{eqn:Einstein}
\end{equation}

Curabitur mi sem, pulvinar quis aliquam rutrum. (1) edf (2) ,
$\Omega=[-1,1]^3$, maecenas leo est, ornare at. $z=-1$ edf $z=1$ sed
interdum felis dapibus sem. $x$ set $y$ ytruem.  Turpis $j$ amet
accumsan enim $y$-lacina; ref $k$-viverra nec porttitor $x$-lacina.

Vestibulum ac diam a odio tempus congue. Vivamus id enim nisi:

\begin{eqnarray}
\cos\bar{\phi}_k Q_{j,k+1,t} + Q_{j,k+1,x}+\frac{\sin^2\bar{\phi}_k}{T\cos\bar{\phi}_k} Q_{j,k+1} &=&\nonumber\\ 
-\cos\phi_k Q_{j,k,t} + Q_{j,k,x}-\frac{\sin^2\phi_k}{T\cos\phi_k} Q_{j,k}\label{edgek}
\end{eqnarray}
and
\begin{eqnarray}
\cos\bar{\phi}_j Q_{j+1,k,t} + Q_{j+1,k,y}+\frac{\sin^2\bar{\phi}_j}{T\cos\bar{\phi}_j} Q_{j+1,k}&=&\nonumber \\
-\cos\phi_j Q_{j,k,t} + Q_{j,k,y}-\frac{\sin^2\phi_j}{T\cos\phi_j} Q_{j,k}.\label{edgej}
\end{eqnarray} 

Donec faucibus purus at tortor egestas eu fermentum dolor
facilisis. Maecenas tempor dui eu neque fringilla rutrum. Mauris
\emph{lobortis} nisl accumsan. Aenean vitae risus ante.
%

\vspace{1cm}
\begin{center}
\begin{tabular}{l l l}
\toprule
\textbf{Treatments}~ & \textbf{Response 1} & \textbf{Response 2}\\
\midrule
Treatment 1 & 0.0003262 & 0.562 \\
Treatment 2 & 0.0015681 & 0.910 \\
Treatment 3 & 0.0009271 & 0.296 \\
\bottomrule
\end{tabular}
\captionof{table}{\color{Green} Table caption}
\end{center}
\vspace{1cm}

Phasellus imperdiet, tortor vitae congue bibendum, felis enim sagittis
lorem, et volutpat ante orci sagittis mi. Morbi rutrum laoreet
semper. Morbi accumsan enim nec tortor consectetur non commodo nisi
sollicitudin. Proin sollicitudin.

\section{Conclusions || Discussion }

O manual do \TeX~\cite{knuth1986} pode ser usado para aprendê-lo, e o livro do
Lamport~\cite{lamport1994} para aprender o \LaTeX, mas se quiser ir a fundo tem
que ver como o \TeX quero o os parágrafos em linhas~\cite{knuth1981}.

Two typesLorem ipsum dolor sit amet, consectetur adipiscing elit, sed
do eiusmod tempor incididunt ut labore et dolore magna aliqua. Uore eu
fugiat nulla pariatur. Excepteur sint occaecat cupidatat non proident,
sunt in culpa qui officia deserunt mollit anim id est laborum.



\begin{itemize}
\item Pellentesque eget orci eros. Fusce ultricies, tellus et
  pellentesque fringilla, ante massa luctus libero, quis tristique
  purus urna nec nibh. 
\item Vestibulum sem ante, hendrerit a gravida ac, blandit quis magna.
\end{itemize}

\bibliographystyle{plain} % Plain referencing style
\bibliography{refs} % Use the example bibliography file %sample.bib

%----------------------------------------------------------------------------------------
%   AGRADECIMENTOS
%----------------------------------------------------------------------------------------
\section*{Acknowledgements}
\small \textnormal{This work is part of the FoTran project, funded by the European Research Council (ERC) under the EU's Horizon 2020 research and innovation program (agreement \textnumero{}~771113). ~We ~also ~thank ~the CSC- IT Center for Science Ltd., for computational resources and NVIDIA AI Technology Center (NVAITC) for the expertise in distributed training.}

\end{multicols}

\end{document}